\input{pre/themainpreumble.tex}
\usepackage{xcolor}
\usepackage{float}
\usepackage{hyperref}
\usepackage{tikz}
\usepackage{pgfplots}
\hypersetup{unicode=true}
\definecolor{linkcolor}{HTML}{FF7800} % цвет ссылок
\definecolor{urlcolor}{HTML}{FF7800} % цвет гиперссылок

\hypersetup{pdfstartview=FitH,  linkcolor=linkcolor,urlcolor=urlcolor, colorlinks=true}
\begin{document}
	\sloppy
	\def\authors{Есюнин М.В., Есюнин Д.В.}
	\def\labnum{}
	\def\labname{Генератор гармонических колебаний}
	\def\sciadviser{Орлов И.Я.}
	% \renewcommand{\contentsname}{Оглавление}
	% \renewcommand{\figurename}{Рис.}
	\renewcommand{\vec}{\mathbf}
	\renewcommand{\phi}{\varphi}
	\renewcommand{\kappa}{\varkappa}
	% нормальный вид вектора, фи и каппа
	\renewcommand{\Re}{\operatorname{Re}}
	\renewcommand{\Im}{\operatorname{Im}}
	% меняем номрмальное начертание Re and Im
	\input{pre/titlepage}
	\begin{spacing}{1.2}
		\tableofcontents
	\end{spacing}
	
	\newpage

\section{Назначение автогенератора}
Данное устройство служит источником незатухающих колебаний синусоидальной формы с управляемой частотой.

\section{Области использования}
Автогенераторы находят широкое применение в радиотехнике (возбудители в радиопередатчиках, гетеродины в радиоприемниках), в измерительной технике (задающие генераторы в генераторах стандартных сигналов, опорные генераторы в схемах автоподстройки частоты), в устройствах автоматики и электронной техники (например, в электронных часах) и т.д.
К наиболее важным техническим характеристикам автогенераторов относятся: диапазон рабочих частот, стабильность и точность выставления частоты, уровень гармоник в спектре выходного сигнала, уровень выходного сигнала.

\section{Принцип работы схемы автогенератора}
Любой автогенератор представляет собой нелинейное устройство, преобразующее энергию источника питания (источника постоянного напряжения) в энергию колебаний. При широком разнообразии известных схем автогенераторов каждая из них, помимо источника питания, должна иметь усилитель и цепь обратной связи. Поэтому в обобщенном виде схема автогенератора (см. рис.\ref{}) содержит четырехполюсник в прямой цепи, соответствующий резонансному усилителю, и четырехполюсник в обратной цепи. (Обратите внимание на взаимное расположение входов и выходов четырехполюсников).
Простейшая схема автогенератора (схема с трансформаторной обратной связью), где в качестве активного элемента резонансного усилителя использован транзистор, приведена на рис.\ref{}. На рис.\ref{} пунктиром выделен четырехполюсник обратной связи.
При изучении автогенератора первостепенное значение имеют два вопроса:
\begin{enumerate}
\item Механизм и условия возникновения колебаний.
\item Существование стационарных колебаний и их устойчивость
\end{enumerate}

\section{Самовозбуждение автогенератора}
При включении в схему автогенератора (рис.\ref{}) источников питания и при выполнении далее обсуждаемых условий в схеме возникают автоколебания. Механизм их возникновения заключается в следующем. Скачек напряжения на коллекторе приведет к быстрому изменению выходного тока транзистора, что вызовет ударное возбуждение резонансного контура. Возникшие в контуре колебания через трансформаторную связь проникают на базу транзистора и вызовут переменную составляющую выходного тока. При соответствующих условиях этот ток будет в фазе с током в резонансном контуре, и в результате возникшие за счет скачка напряжения питания собственные колебания в контуре могут со временем не только не ослабевать, но и усиливаться. По мере увеличения уровня колебаний все в большей степени будет проявляться нелинейность характеристик транзистора, что, в свою очередь, приведет к снижению скорости нарастания колебаний в контуре, а затем и к прекращению их роста - колебания приобретают стационарный характер.

При возникновении автоколебаний их уровень на некотором начальном интервале времени остается весьма малым. По этой причине при обсуждении условий самовозбуждения можно пользоваться линейной моделью в виде двух линейных четырехполюсников, соединенных по схеме рис.\ref{}. Обозначим через $K_1(\omega)$ и $K_2(\omega)$ комплексные коэффициенты передачи четырехполюсников прямой и обратной цепи соответственно
\begin{equation*}
\begin{aligned*}
&\dot{K}_1(\omega)=\frac{\dot{U}_{\text{вых}1}}{\dot{U}_{\text{вх}1}} \\
&\dot{K}_2(\omega)=\frac{\dot{U}_{\text{вых}2}}{\dot{U}_{\text{вх}2}} 
\end{aligned*}
\end{equation*}
где $\dot{U}$ - комплексная амплитуда.

Перерисуем схему рис.\ref{} в более удобном для обсуждения виде (см.рис.\ref{}). Легко заметить, что при $\dot{U}_\text{вх}$, схемы на рис.\ref{} и рис.\ref{} совпадают.

Для линейного четырехполюсника (рис.\ref{}) введем комплексный коэффициент передачи $\dot{K}(\omega)$
\begin{equation*}
\dot{K}(\omega)=\frac{\dot{U}_\text{вых}}{\dot{U}_\text{вх}}
\end{equation*}
Поскольку
\begin{equation*}
\begin{aligned}
&\dot{U}_\text{вых}=\dot{U}_{\text{вых}1}=\dot{U}_{\text{вх}1}\dot{K}_1(\omega)=\\
&= [\dot{U}_{\text{вх}1}+\dot{K}_2(\omega)\dot{U}_{\text{вых}1}]\dot{K}_1(\omega)
\end{aligned}
\end{equation*}
то 
\begin{equation*}
\dot{K}(\omega)=\frac{\dot{K}_1(\omega)}{1-\dot{K}_1(\omega)\dot{K}_2(\omega)}
\end{equation*}

Наличие особой точки у комплексной функции $\dot{K}(\omega)$ при условии $1-\dot{K}_1(\omega)\dot{K}_2(\omega)$ физически можно интерпретировать следующим образом: схема на рис.\ref{} при выполнении условия $1-\dot{K}_1(\omega)\dot{K}_2(\omega)$ выдает на выходе колебание с ненулевой амплитудой при бесконечно малой (нулевой) амплитуде колебания на входе. Следовательно, схема  на рис.\ref{} при названных условиях является автогенератором.
Условия самовозбуждения вытекают из равенства
\begin{equation*}
\dot{K}(\omega)=\frac{\dot{K}_1(\omega)}{1-\dot{K}_1(\omega)\dot{K}_2(\omega)}
\end{equation*}

Отсюда 
\begin{equation}
\begin{aligned}
& |\dot{K}_1(\omega)||\dot{K}_2(\omega)|=1 & \phi_1(\omega)+\phi_2(\omega)=2\pi n
\end{aligned}
\label{eq:1}
\end{equation}
где $\phi_1(\omega)$ и $\phi_2(\omega)$ - аргументы функций $\dot{K}_1(\omega)$ и $\dot{K}_2(\omega)$; $n$ - целое число. Первое условие получило название "баланс амплитуд", второе - "баланс фаз".
Равенства \eqref{1} можно рассматривать как уравнения относительно переменной $\omega$. Корни этих уравнений являются теми частотами, на которых возможно возбуждение. Частота генерации - корень системы уравнений \eqref{eq:1}
Таким образом, если в схеме автогенератора на какой-либо частоте $\omega *$ модуль комплексного коэффициента передачи разомкнутого кольца обратной связи $|\dot{K}_1(\omega)||\dot{K}_2(\omega)|$ равен 1, а суммарный набег фаз при прохождении сигнала с этой частотой по тому же кольцу составит $2\pi n$, то в схеме произойдет самовозбуждение. Частота генерируемых колебаний Судет равна $\omega *$.
Выполнение условий самовозбуждения, по существу, означает, что возникшие колебания схемой автогенератора будут поддерживаться на неизменном уровне; неизбежные потери в кольце обратной связи полностью скомпенсированы.
Если условие $|\dot{K}_1(\omega)||\dot{K}_2(\omega)|=1$ не выполнено, имеем
\begin{enumerate}
\item {
	$|\dot{K}_1(\omega)||\dot{K}_2(\omega)|=0$ при $\dot{K}_2(\omega)=0$ - резонансный усилитель. 
}
\item {
	$|\dot{K}_1(\omega)||\dot{K}_2(\omega)|<1$ при $\phi_1(\omega)+\phi_2(\omega)=2\pi n$ - регенеративный усилитель.
}
\item {
	$|\dot{K}_1(\omega)||\dot{K}_2(\omega)|>1$ при $\phi_1(\omega)+\phi_2(\omega)=2\pi n$ - генератор нарастающих колебаний.
}
\end{enumerate}
Рассмотрим каждый из этих случаев

\subsection{Резонансный усилитель}
При малом уровне входного сигнала усилитель работает в линейном режиме: $\dot{K}(\omega)$ является его исчерпывающей характеристикой.
При возрастании амплитуды входного колебания $U_\text{вх}(t)=U_0\cos \omega_0 t$ линейность усилителя будет нарушена. Аппроксимируя проходную динамическую характеристику транзистора $i_k=f(U_\text{вх}$ степенным полиномом степени $N$, выходной ток усилительного каскада можно записать в виде
\begin{equation}
i_\text{вых}=\sum\limits_{n=0}^{N}b_n U^n_\text{вх}
\label{eq:2}
\end{equation}
где $b_n$ - постоянные коэффициенты.
Подставив в \eqref{eq:2} выражение для $U_\text{вх}(t)$, находим амплитуду первой гармоники выходного тока в виде
\begin{equation}
I_1=U_0[b_1+\frac{3}{4}b_3U_0^2+\frac{5}{8}b_5U_0^4+\ldots]
\label{eq:3}
\end{equation}

По аналогии с линейным случаем, где $I_1=S_0U_0$, $S_0$ - крутизна в рабочей точке, для нелинейного усилителя можно записать 
\begin{equation}
I_1=S_\text{ср}(U_0)\cdot U_0
\label{eq:4}
\end{equation}
$S_\text{ср}(U_0)$ - средняя крутизна, которая находится из \eqref{eq:4} в соответствии с определением $S_\text{ср}(U_0)=I_1/U_0$
\begin{equation}
S_\text{ср}(U_0)=b_1+\frac{3}{4}b_3U_0^2+\frac{5}{8}b_5U_0^4+\ldots
\label{eq:5}
\end{equation}

Как следует из \eqref{eq:5}, зависимость $S_\text{ср}(U_0)$ полностью определяется коэффициентом аппроксимирующего полинома $b_{2n+1}$, а сами коэффициенты зависят как от типа электронного прибора и нагрузки, так и от режима его работы.

Чтобы выяснить характер зависимость $S_\text{ср}(U_0)$, рассмотрим рис.\ref{}, где изображены проходная характеристика транзистора и ее крутизна.

Различие рисунков \ref{} и \ref{} состоит в том, что в первом случае с помощью смещения $E_\text{см}$ рабочая точка выбрана на нелинейном участке проходной характеристики, во втором случае - на линейном, где в окрестности рабочей точки $S\approx\text{const}$. При воздействии на усилитель входного синусоидального напряжения с достаточно большой амплитудой $U_0$ крутизна характеристики описывается периодическими функциями времени $S_1(t)$ и $S_2(t)$ , а постоянные составляющие $S'_\text{ср}$ и $S''_\text{ср}$ являются значениями средней крутизны, соответствующими амплитуде $U_0$. Нетрудно заметить, что при увеличении амплитуды входного колебания в случае рис.\ref{} $S_\text{ср}(U_0)$ будет возрастать, в случае рис.\ref{} - падать. На рис. \ref{} представлены два характерных вида зависимости $S_\text{ср}(U_0)$, при этом кривая 1 соответствует рис.\ref{}, кривая 2 - рис.\ref{}.
Зависимость \eqref{eq:4} амплитуды первой гармоники выходного тока $I_1$ от амплитуды колебания на входе $U_0$, получившая название "колебательной характеристики", в соответствии с кривыми на рис.\ref{} так же имеет два характерных вида. На рис.\ref{} - кривая 1 и 2 соответствуют кривым 1 и 2 на рис.\ref{}. Поскольку при настройке контура усилителя на частоту усиливаемого сигнала фаза напряжения на контуре совпадает с фазой первой гармоники тока, то кривые на рис.\ref{} отражают и характер зависимостей $U_k=f(U_0)$
Коэффициент усиления по первой гармонике при работе усилителя в режиме большого сигнала $\dot{K}(\omega_0)$ в соответствии с \eqref{eq:4} и рис.\ref{} является зависимым от $U_0$
\begin{equation}
\dot{K}_1(\omega)=\dot{U}_\text{вых}/\dot{U}_\text{вх}=\dot{U}_k/\dot{U}_\text{вх}=f(U_0)
\label{eq:6}
\end{equation}

\subsection{Ренегеративный усилитель}
При положительной обратной связи в усилителе, т.е. при $\phi_1(\omega)+\phi_2(\omega)=2\pi n$, при $0<|\dot{K}_1(\omega)||\dot{K}_2(\omega)|<1$ автоколебания в схеме на рис.\ref{} отсутствуют, а сама она представляет собой регенеративный усилитель. В радиотехнике под регенерацией подразумевается частичная компенсация потерь в колебательной системе с помощью положительной обратной связи. Явление регенерации позволяет повысить коэффициент усиления усилителя и его избирательность. Компенсация потерь увеличивает добротность контура. На рис.\ref{} иллюстрируется влияние степени связи(т.е. величины $|\dot{K}_2(\omega)|$) на усиления и избирательность.

Наряду с отмеченными положительными свойствами регенеративного усилителя, ему свойственен и существенный недостаток - опасность возбуждения усилителя за счет случайных изменений $|\dot{K}_1(\omega)|$.
\subsection{Ограничение наростающих колебаний. Стационарный режим генератора} 
Строго говоря, выполнение условия $\dot{K}_1(\omega)\dot{K}_2(\omega)=1$, при $\phi_1(\omega)+\phi_2(\omega)=2\pi n$ означает лишь способность схемы на рис.\ref{} поддерживать незатухающие колебания, если они возникнут в ней за счет какого-либо внешнего воздействия. Для того, чтобы автоколебания достигли некоторого наперед заданного уровня необходимо обеспечить им нарастающий характер, что соответствует условию
\begin{equation*}
|\dot{K}_1(\omega)\dot{K}_2(\omega)|>1
\end{equation*}

По мере роста амплитуды колебаний все в болышей мере будет проявляться нелинейность усилителя в прямой цепи. При этом средняя крутизна в соответствии с рис.\ref{} будет уменьшаться, снижая $\dot{K}_1(\omega)$. Снижение $\dot{K}_1(\omega)$, в конечном итоге, приведет к тому, что будет выполнено условие
\begin{equation*}
\dot{K}_1(\omega)\dot{K}_2(\omega)=1
\end{equation*}

На этом рост амплитуды колебаний прекратится: переходный режим завершится, наступит стационарный режим автогенератора.

Определение стационарной амплитуды колебаний удобно проводить с использованием колебательной характеристики (рис.\ref{})
На рис.\ref{} в одной системе координат представлены две зависимости (см.рис\ref{}) 
\begin{equation*}
U_{\text{вых}1}=K_1(\omega_0)U_{\text{вх}1}
\end{equation*}
- колебательная характеристика (кривая 1)
\begin{equation*}
U_{\text{вх}2}=\frac{1}{K_2(\omega_0)}U_{\text{вых}2}
\end{equation*}
или
\begin{equation*}
U_{\text{вых}1}=\frac{1}{K_2(\omega_0)}U_{\text{вх}1}
\end{equation*}
- прямая обратной связи (кривая 2).
Точка пересечения кривых 1 и 2 (точка $а$) означает
\begin{equation*}
K_1(\omega_0)=\frac{1}{K_2(\omega_0)}
\end{equation*}
или
\begin{equation}
K_1(\omega_0)K_2(\omega_0)=1
\label{eq:7}
\end{equation}
т.е. соответствует стационарной амплитуде автоколебаний.

Отметим, что точка $O$ тоже удовлетворяет условию \eqref{eq:7} и соответствует второму стационарному состоянию. Убедимся, что точка a соответствует устойчивому стационарному состоянию, $а$ точка $O$ - неустойчивому.

Пусть схема находится в точке $O$. Если флуктуация приведет к амплитуде $Р$ напряжения база-эмиттер, то амплитуда напряжения на контуре будет $R$: по обратной связи это вызовет увеличение амплитуды напряжения база-эмиттер до величины $S$, что, в свою очередь, вызовет переход в точку $T$ и т.д., пока схема не придет к точке $a$.

Проведем аналогичные рассуждения относительно состояния в точке $a$. Пусть флуктуация выведет амплитуду напряжения на контуре из точки $a$ в точку $b$. Через обратную связь (через точку $c$) это вызовет амплитуду напряжения база-эмиттер величиной $d$, но ей будет соответствовать амплитуда $e$ напряжения на контуре. Другими словами, схема вернется в состояние $a$, что и доказывает устойчивость этого состояния.

Совершенно аналогичным путем легко доказать устойчивость состояний $O$ и $a$ и неустойчивость состояния $b$ для схемы, имеющей иной вид колебательной характеристики (рис.\ref{}).

Режим возбуждения автогенератора, проиллюстрированный рис.\ref{}, называют мягким, режим, соответствующий рис.\ref{} - жестким режимом возбуждения. Различие между мягким и жестким режимами возбуждения, выявляемое при сравнении рис.\ref{} и рис.\ref{}, наглядно прослеживается и в характере зависимости амплитуды стационарных колебаний от степени связи, т.е. от величины представленной на рис.\ref{} для мягкого режима и на рис.\ref{} - для жесткого.

Наличие петли гистерезиса на рис.\ref{} объясняется тем, что колебания возникают при связи, большей, чем связь, при которой происходит срыв колебаний. Это обстоятельство становится ясным из рис.\ref{}: колебания возбуждаются при связи $K'_2(\omega_0) $, а срываются при $K''_2(\omega_0)<K'_2(\omega_0)$.

Следует заметить, что для возникновения колебаний в автогенераторе с жестким режимом возбуждения необходим внешний толчек, достаточный, чтобы вывести схему вверх через порог, задаваемый точкой $b$ (см.рис.\ref{}).

\section{Анализ схемы автогенератора}
Существует множество различных вариантов технической реали¬зации автогенератора.
Простейшая схема автогенератора с индуктивной обратной связью, где в качестве усилительного элемента использован тран¬зистор, приведена на рис.\ref{}. Здесь избирательность по частоте обеспечивается параллельным колебательным контуром, включенным в коллекторную цепь транзистора $T$.

Колебательный контур, собственные потери которого характеризуются сопротивлением $r$, на резонансной частоте $\omega_0=1/{LC}$ имеет сопротивление $R=\rho^2/r$, где $\rho=\sqrt{L/C}$. Добротность контура $Q=\rho/r\gg1$

Для анализа процессов, происходящих в генераторе, воспользуемся его эквивалентной схемой по переменному току, изображенной на риc.\ref{}. Коллекторный ток
\begin{equation*}
\begin{aligned}
&i_k=i_C+i_R+i_L \\
&i_C=C\frac{\operatorname dU_k}{\operatorname dt} \\
&i_L=\frac{1}{L}\int U_k \operatorname dt \\
&i_r=\frac{U_k}{R}
\end{aligned}
\end{equation*}
- соответственно ток через емкость, сопротивление и индуктивность колебательного контура.

Если рассматривать ту область частот,где инерционными свойствами транзистора, т.е. зависимостью его параметров от частоты,можно пренебречь,то ток коллектора в зависимости от напряжений на базе $U_\text{б}$ на коллекторе $U_k$ транзистора можно представить в виде функции $i_k(t)=i_k(U_\text{б}(t),U_k(t))$. Приемлимой аппроксимацией является представление этой функции в виде $i_k=i_k(U_\text{б}-DU_k)$, когда $i_k$ зависит не от каждого из напряжений $U_\text{б}$ и $U_k$ в отдельности, а от управляющего напряжения $U_\text{упр}=U_\text{б}-DU_k$. Параметр $D$, называемый проницаемостью, характеризует влияние коллекторного напряжения на выходной ток транзистора. С учетом сказанного выше
\begin{equation}
i_k=i_k(U_\text{б}-DU_k)=C\frac{\operatorname dU_k}{\operatorname dt}+\frac{U_k}{R}+\frac{1}{L}\int U_k \operatorname dt
\label{eq:8}
\end{equation}{}

В пренебрежении током базы напряжение $\displaystyle U_\text{б}=М\frac{\operatorname di_L}{\operatorname dt}$, а $\displaystyle U_k=L\frac{\operatorname di_L}{\operatorname dt}$. Отсюда следует, что $U_\text{упр}=U_\text{б}-DU_k=(M/L-D)U_k=\kappa U_k$. Продифференцировав \eqref{eq:8} по времени, получаем следующее нелинейное дифференциальное уравнение
\begin{equation}
\frac{\operatorname d^2U_k}{\operatorname dt^2}+\frac{\operatorname d}{\operatorname dt}[\frac{U_k}{CR}-\frac{1}{C}i_k(\kappa U_k)]+\omega_0^2U_k=0
\label{eq:9}
\end{equation}

Для его решения необходимо знать конкретную зависимость $i_k(\kappa U_k)$, которая выше описана степенным полиномом \eqref{eq:2}. 

При отсутствии внешних возмущений колебания в генераторе возникнут, когда будут выполнены условия его самовозбуждения. В этом случае выходное напряжение сначала будет нарастать со временем, а
затем выйдет на стационарный уровень с постоянной амплитудой $U_\text{ст}$ (рис.\ref{}). Найдем
\begin{enumerate}
\item условия возникновения колебаний в автогенераторе
\item стационарную амплитуду автоколебаний.
\end{enumerate}

Рассмотрим начальную стадию процесса генерации для времен много меньших времени установления колебаний $t_\text{уст}$. В этом случае уровень колебаний незначителен и транзистор находится в линейном режиме. В разложении $i_k=i_k(\kappa U_k)$ по степеням $\kappa U_k$ отличным от нуля будет лишь коэффициент $b_1=S_0$, остальные $b_n=0$ ($n\gg2$).

Тогда вместо уравнения \eqref{eq:9} получаем линейное дифференциальное уравнение с постоянными коэффициентами
\begin{equation}
\frac{\operatorname d^2U_k}{\operatorname dt^2}+2\alpha \frac{\operatorname d U_k}{\operatorname dt}+\omega_0^2U_k=0
\label{eq:10}
\end{equation}
в котором
\begin{equation}
2\alpha=\frac{1}{L}(r+\frac{\rho^2}{r_k}-\frac{S_0M}{C})
\label{eq:11}
\end{equation}
$\displaystyle r=\rho^2/R$ - собственное активное сопротивление колебательного контура, $\displaystyle \rho^2/r_k=r_\text{вн}$ - внесенное в контур сопротивление за счет шунтирующего действия на него внутреннего сопротивления транзистора $r_k$; $-S_0M/C=r_{-}$ - добавочное сопротивление, вносимое в контур за счет обратной связи. 

Общее решение уравнения \eqref{eq:10}

\end{document}