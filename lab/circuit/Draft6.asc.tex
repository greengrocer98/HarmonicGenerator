% automatically generated document using lt2circuiTikz
\documentclass[tikz,margin={2pt 2pt 2pt 2pt}]{standalone}
\usepackage[T2A]{fontenc}
\usepackage[utf8x]{inputenc}
\usepackage[compatibility,siunitx,  americanvoltages, americancurrents, europeanresistors, europeaninductors, americanports,%
  straightlabels, fetbodydiode, straightvoltages]{circuitikz}
\usepackage{tikz,amsmath, amssymb,bm,color,pgfkeys,siunitx,ifthen,ulem}
\usepackage{pgfplots}
\pgfplotsset{compat=1.14}
\usetikzlibrary{shapes,arrows}
%\usepackage{agaramondc}					% Adobe Garamond, custom shape
%\renewcommand{\shapedefault}{rtl} % rtl: roman tabular lining

\makeatletter

%% bandstop filter (adapted from highpass)
\pgfcircdeclarebipole{}{\ctikzvalof{bipoles/highpass/width}}{*bandstop}{\ctikzvalof{bipoles/highpass/width}}{\ctikzvalof{bipoles/highpass/width}}{
	\pgf@circ@res@step = \ctikzvalof{bipoles/highpass/width}\pgf@circ@Rlen
	\divide \pgf@circ@res@step by 2
	
	\pgfpathmoveto{\pgfpoint{\pgf@circ@res@left}{\pgf@circ@res@zero}}
	\pgf@circ@res@other = \pgf@circ@res@left
	\advance\pgf@circ@res@other by \pgf@circ@res@step 
	
	\ifpgf@circuit@dashed
	\pgfsetdash{{0.1cm}{0.1cm}}{0cm} 
	\fi	
	
	% draw outer box
	\pgfsetlinewidth{\pgfkeysvalueof{/tikz/circuitikz/bipoles/thickness}\pgfstartlinewidth}
	\pgfpathrectanglecorners{\pgfpoint{\pgf@circ@res@left}{\pgf@circ@res@up}}{\pgfpoint{\pgf@circ@res@right}{\pgf@circ@res@down}}
	\pgfusepath{draw}
	
	\ifpgf@circuit@inputarrow
	{
		\advance \pgf@circ@res@left by -.5\pgfkeysvalueof{/tikz/circuitikz/bipoles/thickness}\pgfstartlinewidth
		\pgftransformshift{\pgfpoint{\pgf@circ@res@left}{0pt}}
		\pgfnode{inputarrow}{tip}{}{pgf@inputarrow}{\pgfusepath{fill}}
	}
	\fi
	
	% rotate inner symbol
	\def\pgfcircmathresult{\expandafter\pgf@circ@stripdecimals\pgf@circ@direction\pgf@nil}
	\ifnum \pgfcircmathresult > 45 \ifnum \pgfcircmathresult < 135
	\pgftransformrotate{270}
	\fi\fi
	\ifnum \pgfcircmathresult > 134 \ifnum \pgfcircmathresult < 225  % 134 degree, because >= 135 is not possible
	\pgftransformrotate{180}
	\fi\fi
	\ifnum \pgfcircmathresult > 224 \ifnum \pgfcircmathresult < 315
	\pgftransformrotate{90}
	\fi\fi
	
	% draw inner symbol
	\pgfsetdash{}{0pt}	% always draw solid line for inner symbol
	\pgfsetarrows{-} %never draw arrows
	\pgfsetlinewidth{\pgfstartlinewidth}
	\pgfpathmoveto{\pgfpoint{-0.5\pgf@circ@res@step}{0.5\pgf@circ@res@step}}
	\pgfpathsine{\pgfpoint{.25\pgf@circ@res@step}{.25\pgf@circ@res@step}}
	\pgfpathcosine{\pgfpoint{.25\pgf@circ@res@step}{-.25\pgf@circ@res@step}}
	\pgfpathsine{\pgfpoint{.25\pgf@circ@res@step}{-.25\pgf@circ@res@step}}
	\pgfpathcosine{\pgfpoint{.25\pgf@circ@res@step}{.25\pgf@circ@res@step}}
	\pgfusepath{draw}
	
	\pgfpathmoveto{\pgfpoint{-0.5\pgf@circ@res@step}{0}}
	\pgfpathsine{\pgfpoint{.25\pgf@circ@res@step}{.25\pgf@circ@res@step}}
	\pgfpathcosine{\pgfpoint{.25\pgf@circ@res@step}{-.25\pgf@circ@res@step}}
	\pgfpathsine{\pgfpoint{.25\pgf@circ@res@step}{-.25\pgf@circ@res@step}}
	\pgfpathcosine{\pgfpoint{.25\pgf@circ@res@step}{.25\pgf@circ@res@step}}
	\pgfusepath{draw}
	\pgfpathmoveto{\pgfpoint{-0.15\pgf@circ@res@step}{-0.15\pgf@circ@res@step}}
	\pgfpathlineto{\pgfpoint{0.15\pgf@circ@res@step}{0.15\pgf@circ@res@step}}
	\pgfusepath{draw}
	
	\pgfpathmoveto{\pgfpoint{-0.5\pgf@circ@res@step}{-0.5\pgf@circ@res@step}}
	\pgfpathsine{\pgfpoint{.25\pgf@circ@res@step}{.25\pgf@circ@res@step}}
	\pgfpathcosine{\pgfpoint{.25\pgf@circ@res@step}{-.25\pgf@circ@res@step}}
	\pgfpathsine{\pgfpoint{.25\pgf@circ@res@step}{-.25\pgf@circ@res@step}}
	\pgfpathcosine{\pgfpoint{.25\pgf@circ@res@step}{.25\pgf@circ@res@step}}
	\pgfusepath{draw}
	%	\pgfpathmoveto{\pgfpoint{-0.15\pgf@circ@res@step}{-0.65\pgf@circ@res@step}}
	%	\pgfpathlineto{\pgfpoint{0.15\pgf@circ@res@step}{-0.35\pgf@circ@res@step}}
	%	\pgfusepath{draw}
}

\tikzset{
	*bandstop/.style={\circuitikzbasekey, /tikz/to path=\pgf@circ@*bandstop@path},
}
\def\pgf@circ@*bandstop@path#1{\pgf@circ@bipole@path{*bandstop}{#1}}




\makeatother

\usetikzlibrary{backgrounds,calc,positioning}

\usetikzlibrary{circuits.ee.IEC}
\usetikzlibrary{arrows}


% sym32a style

\ctikzset{tripoles/mos style/arrows}
\ctikzset{
	/tikz/circuitikz/quadpoles/coupler/width=1,%1.3
	/tikz/circuitikz/quadpoles/coupler/height=0.952,%1.3
	/tikz/circuitikz/quadpoles/coupler2/width=1,%1.3
	/tikz/circuitikz/quadpoles/coupler2/height=0.952,%1.3
	/tikz/circuitikz/quadpoles/transformer/width=1.425,%1.5
	/tikz/circuitikz/quadpoles/transformer/height=1.425,%1.5
	/tikz/circuitikz/quadpoles/transformer core/width=1.425,%1.5
	/tikz/circuitikz/quadpoles/transformer core/height=1.425,%1.5
	/tikz/circuitikz/quadpoles/gyrator/width=1.425,%1.5
	/tikz/circuitikz/quadpoles/gyrator/height=1.425,%1.5
	/tikz/circuitikz/monopoles/tlinestub/width=0.1875,%0.25 no effect!
	/tikz/circuitikz/tripoles/american and port/height=0.95,%.8
	/tikz/circuitikz/tripoles/american nand port/height=0.95,%.8
	/tikz/circuitikz/tripoles/american or port/height=0.95,%.8
	/tikz/circuitikz/tripoles/american nor port/height=0.95,%.8
	/tikz/circuitikz/tripoles/american xor port/height=0.95,%.8
	/tikz/circuitikz/tripoles/american xnor port/height=0.95,%.8
	/tikz/circuitikz/bipoles/tline/height=0.4,%0.3
%	/tikz/circuitikz/bipoles/tline/width=1.2,%0.8
	/tikz/circuitikz/bipoles/diode/height=0.375,%
	/tikz/circuitikz/bipoles/diode/width=0.375,%
	/tikz/circuitikz/bipoles/varcap/height=0.375,%
	/tikz/circuitikz/bipoles/varcap/width=0.375,%
	/tikz/circuitikz/tripoles/triac/height=1.05,%
	/tikz/circuitikz/tripoles/triac/width=0.952,%
	/tikz/circuitikz/tripoles/thyristor/height=1.05,%
	/tikz/circuitikz/tripoles/thyristor/width=0.952,%
	/tikz/circuitikz/tripoles/op amp/height=0.952,%
	/tikz/circuitikz/tripoles/op amp/width=1.2,%
	/tikz/circuitikz/tripoles/op amp/font=\footnotesize,
	/tikz/circuitikz/tripoles/gm amp/height=0.952,% 1.7
	/tikz/circuitikz/tripoles/gm amp/width=1.2,% 1.4
	%	/tikz/circuitikz/tripoles/gm amp/font=\footnotesize,
	/tikz/circuitikz/tripoles/plain amp/height=0.952,% 1.7
	/tikz/circuitikz/tripoles/plain amp/width=1.2,% 1.4
	/tikz/circuitikz/bipoles/resistor/voltage/straight label distance/.initial=.8,
	/tikz/circuitikz/bipoles/generic/voltage/straight label distance/.initial=.8,
	/tikz/circuitikz/bipoles/inductor/voltage/straight label distance/.initial=.8,
	/tikz/circuitikz/bipoles/fullgeneric/voltage/straight label distance/.initial=.8,
	/tikz/circuitikz/bipoles/capacitor/voltage/straight label distance/.initial=1.0,
	/tikz/circuitikz/bipoles/thickness=1.6,
}
\ctikzset{v/.append style={/tikz/european voltages}}

\definecolor{netlabelcolor}{rgb}{0, 0, 0.25}
\definecolor{lttotitextcolor}{rgb}{0, 0, 0}
\definecolor{lttotidrawcolor}{rgb}{0, 0, 0}
\definecolor{netcolor}{rgb}{0, 0, 0}

\pgfkeys{/lt2ti/netlabel/font/.initial= \small}
\pgfkeys{/lt2ti/text/font/.initial= \small}

\pgfkeys{/lt2ti/Net/.style= {netcolor}}
\tikzstyle{dashdotdotted}=[dash pattern=on 3pt off 2pt on \the\pgflinewidth off 2pt on \the\pgflinewidth off 2pt]

\pgfkeys{/lt2ti/VArrow/.style= {<->,>=latex}}
\pgfkeys{/lt2ti/SArrow/.style= {->,>=angle 90}}

\begin{document}%
	%\centering%
		\begin{tikzpicture}[circuit ee IEC, scale=1]% default: 0.4
	%\tikzstyle{every node}=[font=\small];%
	%\node [draw] at (0.0,0.0) {\pgfkeysvalueof{/tikz/circuitikz/tripoles/op amp/font}};
\draw [/lt2ti/Net](-9.0,19.0)to[*short,-, color=netcolor] (-9.0,19.0);% wire w4_w3_w8 start
\draw [/lt2ti/Net](2.0,18.5)to[*short,-*, color=netcolor] (2.0,18.5);% wire w4_w3_w8 end
\draw [/lt2ti/Net](-9.0,19.0) --  (-9.0,20.0) --  (2.0,20.0) -- (2.0,18.5); % wire w4_w3_w8 polyline 
\draw [/lt2ti/Net](-5.0,18.5)to[*short,*-, color=netcolor] (-5.0,18.5);% wire w14_w5_w13 start
\draw [/lt2ti/Net](-7.5,18.0)to[*short,-, color=netcolor] (-7.5,18.0);% wire w14_w5_w13 end
\draw [/lt2ti/Net](-5.0,18.5) --  (-7.0,18.5) --  (-7.0,18.0) -- (-7.5,18.0); % wire w14_w5_w13 polyline 
\draw [/lt2ti/Net](-4.0,18.5)to[*short,*-*, color=netcolor] (-5.0,18.5);% wire w6
\draw [/lt2ti/Net](-2.5,18.5)to[*short,-*, color=netcolor] (-4.0,18.5);% wire w7
\draw [/lt2ti/Net](2.0,18.5)to[*short,*-, color=netcolor] (-1.0,18.5);% wire w9
\draw [/lt2ti/Net](3.5,18.5)to[*short,*-*, color=netcolor] (2.0,18.5);% wire w10
\draw [/lt2ti/Net](5.0,18.5)to[*short,*-*, color=netcolor] (3.5,18.5);% wire w11
\draw [/lt2ti/Net](5.5,18.5)to[*short,-*, color=netcolor] (5.0,18.5);% wire w12
\draw [/lt2ti/Net](-9.0,16.0)to[*short,*-, color=netcolor] (-9.0,17.0);% wire w15
\draw [/lt2ti/Net](-7.5,16.0)to[*short,*-*, color=netcolor] (-9.0,16.0);% wire w16
\draw [/lt2ti/Net](-5.0,16.0)to[*short,*-*, color=netcolor] (-6.5,16.0);% wire w17
\draw [/lt2ti/Net](-4.0,16.0)to[*short,*-*, color=netcolor] (-5.0,16.0);% wire w18
\draw [/lt2ti/Net](-2.5,16.5)to[*short,-, color=netcolor] (-2.5,16.5);% wire w19_w20 start
\draw [/lt2ti/Net](-4.0,16.0)to[*short,-*, color=netcolor] (-4.0,16.0);% wire w19_w20 end
\draw [/lt2ti/Net](-2.5,16.5) --  (-2.5,16.0) -- (-4.0,16.0); % wire w19_w20 polyline 
\draw [/lt2ti/Net](-1.0,16.0)to[*short,-, color=netcolor] (-1.0,16.5);% wire w21
\draw [/lt2ti/Net](2.0,16.0)to[*short,*-, color=netcolor] (1.0,16.0);% wire w22
\draw [/lt2ti/Net](2.0,16.0)to[*short,*-, color=netcolor] (2.0,16.0);% wire w29_w30 start
\draw [/lt2ti/Net](-6.5,15.0)to[*short,-*, color=netcolor] (-6.5,15.0);% wire w29_w30 end
\draw [/lt2ti/Net](2.0,16.0) --  (2.0,15.0) -- (-6.5,15.0); % wire w29_w30 polyline 
\draw [/lt2ti/Net](3.5,16.0)to[*short,*-, color=netcolor] (3.5,16.5);% wire w23
\draw [/lt2ti/Net](3.5,16.0)to[*short,*-*, color=netcolor] (2.0,16.0);% wire w24
\draw [/lt2ti/Net](5.0,16.0)to[*short,*-*, color=netcolor] (3.5,16.0);% wire w25
\draw [/lt2ti/Net](5.5,16.0)to[*short,-*, color=netcolor] (5.0,16.0);% wire w26
\draw [/lt2ti/Net](-9.0,15.0)to[*short,*-*, color=netcolor] (-9.0,16.0);% wire w27
\draw [/lt2ti/Net](-7.5,15.0)to[*short,*-*, color=netcolor] (-9.0,15.0);% wire w28
\draw [/lt2ti/Net](-9.0,14.0)to[*short,-*, color=netcolor] (-9.0,15.0);% wire w31
 \draw (-9.0, 18.0) node[npn, nobodydiode, xscale=-1, rotate=0, ] (Q1) {}   (Q1)++(1.0,1) node {T}; % component "circuiTikz\\npn_t" "Q1" 
 \draw [/lt2ti/Net](-7.5, 18.0) to [*short, -] (Q1.B); \draw [/lt2ti/Net](-9.0, 17.0) to [*short, -] (Q1.E); \draw [/lt2ti/Net](-9.0, 19.0) to [*short, -] (Q1.C);% extend wires to the connection points   % component "circuiTikz\\npn_t" "Q1" 
  \draw (-2.5, 18.5) to[*cute inductor, , a_=, -, , ] (-2.5,16.5){};  %\node [] at (-3.0,19.0) {x}; % component "circuiTikz\\ind_coil" "L1" 
 \draw (-9.0, 14.0) node[rground, xscale=1, yscale=1, rotate=0, ] (X1) {};%  (X1)++(0.0,0.0) node {X1 }; % component "circuiTikz\\gnd" "X1" 
  \draw (-1.0, 16.5) to[*cute inductor, l^=$L$, a_=, -, , ] (-1.0,18.5){};  %\node [] at (-0.5,16.0) {x}; % component "circuiTikz\\ind_coil" "L2" 
  \draw (1.0, 16.0) to[*resistor, l_=$r$, a_=, -, ] (-1.0,16.0){}; %\node [] at (0.0,16.0) {x}; % component "circuiTikz\\res_eu" "R1" 
  \draw (3.5, 18.5) to[*capacitor, l^=$C$, a_=, *-, ] (3.5,16.5){}; % component "cap" "C1" 
  %\node [] at (3.0,18.5) {x}; % component "cap" "C1" 
  \draw [/lt2ti/VArrow] (-5.0, 16.0) -- (-5.0,18.5) node [midway, left] {$U_\text{{вх}1}$ }; % volt.arrow % component "circuiTikz\\arrow_voltage_5" "U1" 
  \draw [/lt2ti/VArrow] (-4.0, 16.0) -- (-4.0,18.5) node [midway, left] {$U_\text{{вых}2}$}; % volt.arrow % component "circuiTikz\\arrow_voltage_5" "U2" 
  \draw [/lt2ti/VArrow] (2.0, 16.0) -- (2.0,18.5) node [midway, right] {$U_\text{{вх}2}$}; % volt.arrow % component "circuiTikz\\arrow_voltage_5" "U3" 
  \draw [/lt2ti/VArrow] (5.0, 16.0) -- (5.0,18.5) node [midway, right] {$U_\text{{вых}1}$}; % volt.arrow % component "circuiTikz\\arrow_voltage_5" "U4" 
  \draw [dashed] (1.0,15.5) rectangle (-3.5,19.0) node{$L_\text{св}$}; % sch Rect % schLine "SchRect" (32, -496)->(-112, -608) style=2
  \node (lbl60) [] at (-7.5,16.5) {};% text mark % text "" "E1 lbl60 " 
  \node (lbl60txt) [ lttotitextcolor, right= -0.25cm of lbl60, scale=0.5*2.0] {{\pgfkeysvalueof{/lt2ti/text/font}$E_\text{см}$}}; % text "" "E1 lbl60 " 
  \node (lbl61) [] at (-7.5,14.5) {};% text mark % text "" "E2 lbl61 " 
  \node (lbl61txt) [ lttotitextcolor, right= -0.25cm of lbl61, scale=0.5*2.0] {{\pgfkeysvalueof{/lt2ti/text/font}$E_k$}}; % text "" "E2 lbl61 " 

	\end{tikzpicture}
\end{document}
